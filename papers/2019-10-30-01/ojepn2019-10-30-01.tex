\documentclass[twocolumn]{article}
\usepackage[utf8]{inputenc}
\usepackage[T1]{fontenc}
\usepackage[english]{babel}
\usepackage{ifpdf,amsmath,amsthm,amssymb,amsfonts,newtxtext,newtxmath} 
\usepackage{array,graphicx,dcolumn,multirow,hevea,abstract,hanging}
\usepackage[labelfont=sc,textfont=sf]{caption}
\usepackage[hyperfootnotes=false,breaklinks=true]{hyperref} % was dvipdfmx
\urlstyle{rm}
\usepackage[hyphenbreaks]{breakurl}
%\usepackage{natbib} % must come afer hyperfootnotes
%\setlength{\bibsep}{0pt}
\usepackage{booktabs} % \toprule \midrule \bottomrule \cmidrule(lr){a-b}
% define centered and ragged columns:
\newcolumntype{L}[1]{>{\raggedright\arraybackslash }p{#1}} % can use m{}
\newcolumntype{C}[1]{>{\centering\arraybackslash }p{#1}}
\newcolumntype{R}[1]{>{\raggedleft\arraybackslash }p{#1}}
\newcolumntype{d}[1]{D{.}{.}{#1}} % d{3.2} for 3 places on l, 2 on r
\newcommand{\mc}{\multicolumn}
\topmargin=-.3in \oddsidemargin=-.1in \evensidemargin=-.1in \textheight=9in \textwidth=6.8in
\setlength\tabcolsep{1mm}
\setlength\columnsep{5mm}
\setlength\abovecaptionskip{1ex}
\setlength\belowcaptionskip{.5ex}
\setlength\belowbottomsep{.3ex}
\setlength\lightrulewidth{.04em}
\renewcommand\arraystretch{1.2}
\renewcommand{\topfraction}{1}
\renewcommand{\textfraction}{0}
\renewcommand{\floatpagefraction}{.9}
% \renewcommand{\baselinestretch}{1.00} \large\normalsize % for fixing spaces
\widowpenalty=1000
\clubpenalty=1000
\setlength{\parskip}{0ex}
\let\tempone\itemize
\let\temptwo\enditemize
\let\tempthree\enumerate
\let\tempfour\endenumerate
\renewenvironment{itemize}{\tempone\setlength{\itemsep}{0pt}}{\temptwo}
\renewenvironment{enumerate}{\tempthree\setlength{\itemsep}{0pt}}{\tempfour}

%%%%%%%%%%%%%%%%%%%%%%%%%%%%%%%%%%%%%%%%%%%%%%%%%%%%%%%%%%%%%%%%%%%%%
\setcounter{page}{1} % start with first page

\title{Editorial}

\author{
Andy J. Wills\thanks{School of Psychology, Plymouth University, U.K. Email: andy@willslab.co.uk}
}


\date{} % leave empty
\begin{document} % goes here

% fill in short title
\newcommand{\jhead}{Open Journal of Experimental Psychology and Neuroscience, 2020}
\newcommand{\jdate}{Vol.~1}
\pagestyle{myheadings} \markright{\protect\small \jhead, \jdate \hfill EDITORIAL \qquad}
\begin{htmlonly}
\jhead, \jdate, pp.\
\end{htmlonly}
%\begin{latexonly}
\twocolumn[
\vspace{-.3in}
{\small \jhead, \jdate, pp.\ 1--2.}
%\end{latexonly}

\maketitle

%\begin{latexonly}
\vspace{-3mm}
\begin{onecolabstract}
%\end{latexonly}
The Open Journal of Experimental Psychology and Neuroscience is a not-for-profit, fully open-access journal for reproducible research reports, theory papers, comments, early reports, and requests for collaborators. Its goals are to promote scientific rigour, support professional development, and minimise publication costs. We aim to support the publication of the best possible version of your science, regardless of your location, status, or career stage

\smallskip
\noindent
Keywords: editorial
%\begin{latexonly}
\end{onecolabstract}\bigskip
]
%\end{latexonly}

{\renewcommand{\thefootnote}{}
\footnotetext{ % note blank lines above and below acknowledgment

  Funding provided by the Cognition Institute, Plymouth University, U.K.

Copyright: \copyright\ 2019.
The authors license this article under the terms of the
\href{http://creativecommons.org/licenses/by/3.0/}{Creative Commons
  Attribution Share-Alike 4.0 License.}
}}

\saythanks

\setlength{\baselineskip}{12pt plus.2pt}

\section{Principles}

The Open Journal of Experimental Psychology and Neuroscience has four principles, which guide all our activities:

\begin{enumerate}

\item \textbf{Reproducible science}. Others should be able to verify your analysis and attempt to replicate your findings. Empirical studies should be adequately powered, unless clearly identified as an Early Report (see below). Use of third-party preregistration services is encouraged.

\item \textbf{Rapid, rigorous, supportive peer review}. Our goal is to work with you to produce a fair, accurate and succinct account of your research. We will not publish papers that we consider are likely to mislead others. 
  
\item \textbf{Full open access}. All articles in OJEPN are immediately and fully open access.

\item \textbf{Low cost}. OJEPN is part of the Nurture Science Publishing Group, an unregistered charity dedicated to minimising the costs of academic publishing. We will never charge more to publish your article than we need to cover our operating costs. Our Author Publishing Charge is currently £300.

\end{enumerate}

  
\section{Process} 

We employ a two-stage peer-review process, designed to quickly return a decision about the publish-ability of your work, so we can all then focus on publishing the best version of your science. In our two-stage process, you initially submit just your Method and Results sections, along with up to 500 words contextualising your work. If you would rather submit a complete paper at this first stage, perhaps because you have already written it in full, you are also welcome to do this. Your submission is rapidly assessed by an action editor against a good science checklist of common issues (e.g. no power calculation, concluding from a null, etc.), and you receive one of two outcomes within a couple of weeks:

\begin{enumerate}
  
\item \textbf{Invited full submission}. Our intention from this point is to work with you and the reviewers to produce a publishable article. Only in exceptional cases would your paper be rejected after this decision.
  
\item \textbf{Rejection}. Your paper is rejected on the basis of a set of clearly-stated reasons related to scientific rigour. You are welcome to submit a revised version if you feel you can address the issues raised.

\end{enumerate}

To help maintain the objectivity of the review process, we ask that you do not suggest reviewers. Instead, we will seek suitably expert reviewers you have not published with. You can request that we do not approach certain individuals as reviewers. 

\section{Article types}

We accept four different types of article:

\paragraph{Standard paper.} This can be an original study, a replication attempt or a review of a published body of work. There is a 4000-word limit, but we encourage you to aim for less. The word count does not include abstract, references, tables, or figures.

\paragraph{Early report.} Think of an early report somewhat like a conference presentation — you want feedback from the community on a procedure or result, before you’ve totally nailed everything. This encourages dialogue and also establishes precedent. 1500-word limit.

\paragraph{Request for collaboration.} You explain the rationale for a large-scale study requiring assistance. This calls for collaborators and establishes precedent for your work. 1500-word limit.

\paragraph{Commentary.} We actively encourage continued dialogue. If you'd like to make substantive comments on an article published in OJEPN, or elsewhere, here's the place to do it. 1000-word limit.


\section{Submission templates}

We prefer submission in LaTeX format, using our template, because this helps to minimise our production costs, and it allows you to see how your article will look once published. Overleaf provides a convenient web-based system for creating LaTeX documents. If you are unable to use LaTeX format, you may submit using our LibreOffice template, which also works with recent versions of Microsoft Word. Visit \url{https://ojepn.com} for further details.

\section{Author processing charge}

We'll never charge more to publish your article than it takes to cover our running costs. Our APC is currently £300, payable on acceptance of your article. If your paper is not in LaTeX format, and contains more than one simple equation, or more than one simple Table, you may be charged an additional fee for typesetting prior to publication. This fee is currently £100. Thanks to our sponsors, there is some funding you might be able to apply for to cover the APC. Visit \url{https://ojepn.com} for further details.

\section{Preprints}

We strongly encourage (but do not require) authors to upload a preprint of their manuscript at the same time they submit it to OJEPN for consideration. 

\section{Submitting an article}

For all queries regarding submission of articles to OJEPN, please email: editor@ojepn.com


\end{document}

%%% Local Variables:
%%% mode: latex
%%% TeX-master: t
%%% End:
